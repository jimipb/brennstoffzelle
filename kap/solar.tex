\section{Solarzelle}

\subsection*{Füllfaktor}
Das Verhältnis zwischen theoretisch erzielbarer Leistung und der am Maximum Power Point (MPP) erzielten Leistung, wird Füllfaktor genannt. (Abbildung)

\subsection*{Wärmeeinwirkung auf Solarzelle}
Die Anzahl der freien Ladungsträger nimmt im Halbleiter mit der Temperatur zu. Diese Ladungsträger bewirken in der Sperrschicht der Solarzelle einen Diffusionsstrom, der die Leistung der Solarzelle reduziert.
(abgeschrieben: überarbeiten!)

\subsection*{Funktionsprinzip einer Solarzelle}
Durch Einstrahlung von Photonen werden freie Ladungsträger in der Zelle erzeugt. Der p-n-Übergang erzeugt ein inneres elektrisches Feld. Das herausgelöste Elektron und das entstandene Loch werden dadurch in unterschiedliche Richtungen transportiert. Hieraus entsteht ein Strom.\\
Das herauslösen eines Elektrons ist nur möglich, wenn die Energie $E=hv$ größer der Gap-Energie des Halbleitermaterials ist.
